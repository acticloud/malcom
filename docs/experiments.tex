\section{Experiments}
\subsection{Tpch SF10}
% To evaluate our memory footpring estimator mechanism we kept each
As a training set for each query, we randomized every selection point and range,
and produced 200 random versions of the query. As a test set we used the original
query.

In query 19 we observe a misprediction of 70\%. The reason this happens, is that
the MAL algebra of this specific query consists of a lot of merge instructions,
for which we output the sum of the two arguments, which in case of merging similar
variables can lead to an almost 2$\times$ overestimation.
This is the basic reason for the large error observed.
\begin{verbatim}
culprit:
C_187 := algebra.thetaselect(X_178, "DELIVER IN PERSON", "==");                                                                                                                                                                                                                    |
X_191 := bat.mergecand(C_187, C_187);                                                                                                                                                                                                                                              |
X_194 := bat.mergecand(X_191, C_187);
\end{verbatim}

The second most deviant query is 16. The dominant reason why there is an 20\%
error is the misprediction of the join instruction(our approach is kNN based).

\begin{figure}[!htb]
  \centering
  \includegraphics[scale=0.7]{figs/tpch10/mem_error_1-23.pdf}
  \caption{Queries 1-22 memory footprint error}
  \label{fig:tpch10m}
\end{figure}

\subsection{Airtraffic Queries}
%TODO some info about airtraffic...
As a training set for each query, we used 1000 versions of the
same query, randomized on the selection values. As a test set we used the original
query.
\subsubsection{Q04}
Query 4 (Figure \ref{sel:sql4}) selects all flights that have a departure delay greater than 15 minutes.
The training queries are randomized based on the DepDelay bound.
From the prediction perspective, this selection is interesting because
departure delay does not follow a uniform distribution(more like a normal),
thus is more challenging to accurately predict.
Figures  \ref{fig:sel04}, \ref{fig:mem04} show the error percentage for the
selection, and the whole query respectively. The x axis is the number of queries
used as a training set, and the y axis the error \%. The initial selection error
for is 500\%, but as we insert more instructions to the dictionary we get closer
to the initial range, and we are able to make an accurate prediction. After the
first 400 instructions, the error drops to (...)\%.
The per query memory error follows a similar pattern,
but the error is smaller because select instructions represent a relatively
small proportion of the overall query memory usage.

\begin{figure}[t]
\begin{lstlisting}
SELECT "DayOfWeek", COUNT(*) AS "Flights"
FROM ontime
WHERE "DepDelay" > 15
GROUP BY "DayOfWeek"
ORDER BY "DayOfWeek";
\end{lstlisting}
  \caption{Query 4}
  \label{sel:sql4}
\end{figure}


\begin{figure}[!htb]
  \begin{subfigure}[t]{0.5\textwidth}
    \includegraphics[scale=0.4]{figs/airtraffic/airtraffic_sel04_error.pdf}
    \caption{Query 04 select error}
    \label{fig:sel04}
  \end{subfigure}
  \begin{subfigure}[t]{0.5\textwidth}
    \includegraphics[scale=0.4]{figs/airtraffic/airtraffic_q04_memerror.pdf}
    \caption{Query 04 memory footprint}
    \label{fig:mem04}
   \end{subfigure}
\end{figure}

\subsubsection{Q09}

Query 9(Figure \ref{sel:sql09}) selects all flights on a specific date
for which the departure and the arrival delay are greater than 15 minutes.
The memory error(Figure \ref{fig:sel09}) is very small,
due to the very high selectivity of the query,
whereas the selection error(Figure \ref{fig:mem09}) seems to fluctuate a lot
betwwn 20 and 40\%. The reason we believe this is happening is that all three columns
(depdelay, arrdelay, date) are strongly correlated, making it very hard
for the prediction algorithm to converge.

\begin{figure}[!htb]
\begin{lstlisting}[frame=single]
WITH t1 AS (
    SELECT "Origin","CRSDepTime","DepDelay","Dest","ArrDelay","CRSArrTime"
    FROM ontime
    WHERE "DepDelay" > 15 AND "ArrDelay" > 15
      AND "Month" = 3 AND "DayofMonth" = 24 AND "Year" = 2013
)
SELECT t1."Origin" AS "Airport1",
  CAST(AVG(t1."DepDelay") AS DECIMAL(8,2)) AS "AVGDepDelay",
  CAST(AVG(t1."ArrDelay") AS DECIMAL(8,2)) AS "AVGArrDelay", t2."Origin" AS "Airport2",
  CAST(AVG(t2."DepDelay") AS DECIMAL(8,2)) AS "AVGDepDelay2",
  CAST(AVG(t2."ArrDelay") AS DECIMAL(8,2)) AS "AVGArrDelay2", t2."Dest" AS "Airport3"
FROM t1, t1 AS t2
WHERE t1."Dest" = t2."Origin" AND t1."CRSArrTime" < t2."CRSDepTime"
GROUP BY t1."Origin", t2."Origin", t2."Dest"
ORDER BY "AVGDepDelay" DESC,"AVGArrDelay" DESC,"AVGDepDelay2" DESC,"AVGArrDelay2" DESC
\end{lstlisting}
  \caption{Query 09}
  \label{sel:sql09}
\end{figure}

\begin{figure}[!htb]
  \begin{subfigure}[t]{0.5\textwidth}
    \includegraphics[scale=0.4]{figs/airtraffic/airtraffic_sel09_error.pdf}
    \caption{Query 09 select error}
    \label{fig:sel09}
  \end{subfigure}
  \begin{subfigure}[t]{0.5\textwidth}
    \includegraphics[scale=0.4]{figs/airtraffic/airtraffic_q09_memerror.pdf}
    \caption{Query 09 memory footprint}
    \label{fig:mem09}
   \end{subfigure}
\end{figure}

\subsubsection{Q10}

Query 10(Figure \ref{sel:sql10}) contains two point selections,
regarding the origin and the destination of the flight.


\begin{figure}[htb!]
\begin{lstlisting}[frame=single]
ITH t1 AS ( -- flights to ORD
    SELECT "Origin" AS ap, COUNT(*) AS cnt_in
    FROM ontime
    WHERE "Dest" = 'ORD'
    GROUP BY "Origin"
),
t2 AS ( -- flights from ORD
    SELECT "Dest" AS ap, COUNT(*) AS cnt_out
    FROM ontime
    WHERE "Origin" = 'ORD'
    GROUP BY "Dest"
),
t3 AS ( -- merge t1, t2 into one table
    SELECT t1.ap AS ap1, t1.cnt_in, t2.ap AS ap2, t2.cnt_out
    FROM t1 FULL OUTER JOIN t2 ON (t1.ap = t2.ap))
SELECT CASE WHEN ap1 IS NULL THEN ap2 ELSE ap1 END AS "Airport",
       cnt_in AS "InboundFlights", cnt_out AS "OutboundFlights"
FROM t3;
\end{lstlisting}
  \caption{Query 10}
  \label{sel:sql10}
\end{figure}

\begin{figure}[htb!]
  \begin{subfigure}[t]{0.5\textwidth}
    \includegraphics[scale=0.4]{figs/airtraffic/airtraffic_sel10_error.pdf}
    \caption{Query 10 select error}
    \label{fig:sel10}
  \end{subfigure}
  \begin{subfigure}[t]{0.5\textwidth}
    \includegraphics[scale=0.4]{figs/airtraffic/airtraffic_q10_memerror.pdf}
    \caption{Query 10 memory footprint}
    \label{fig:mem10}
   \end{subfigure}
\end{figure}


\subsubsection{Q11}

\begin{figure}[htb!]
\begin{lstlisting}[frame=single]
WITH t1 AS ( -- #flights per route before 9/11
    SELECT SQL_MIN("Origin", "Dest") || ' <-> ' ||
           SQL_MAX("Origin", "Dest") AS route,
           COUNT(*) AS cnt_before
    FROM ontime
    WHERE '2010-09-11' < "FlightDate" AND "FlightDate" < '2011-09-11'
    GROUP BY route
),
t2 AS ( -- #flights per route after 8/11
    SELECT SQL_MIN("Origin", "Dest") || ' <-> ' ||
           SQL_MAX("Origin", "Dest") AS route,
           COUNT(*) AS cnt_after
    FROM ontime
    WHERE '2011-09-11' <= "FlightDate" AND "FlightDate" < '2012-09-11'
    GROUP BY route
),
t3 AS ( -- merge t1, t2 into one table
    SELECT t1.route AS route1, t1.cnt_before, t2.route AS route2, t2.cnt_after
    FROM t1 FULL OUTER JOIN t2 ON (t1.route = t2.route)
)
SELECT CASE WHEN route1 IS NULL THEN route2 ELSE route1 END AS "Route",
       cnt_before AS "FlightsBefore", cnt_after AS "FlightsAfter"
FROM t3;
\end{lstlisting}
  \caption{Query 11}
  \label{sel:sql11}
\end{figure}

\begin{figure}[htb!]
 \begin{subfigure}[t]{0.5\textwidth}
   \includegraphics[scale=0.4]{figs/airtraffic/airtraffic_sel11_error.pdf}
   \caption{Query 11 select error}
   \label{fig:sel11}
 \end{subfigure}
 \begin{subfigure}[t]{0.5\textwidth}
   \includegraphics[scale=0.4]{figs/airtraffic/airtraffic_q11_memerror.pdf}
   \caption{Query 11 memory footprint}
   \label{fig:sel11}
  \end{subfigure}


\end{figure}


\subsubsection{Q15}

\begin{figure}[htb!]
\begin{lstlisting}[frame=single]
WITH t1 AS (
    SELECT "Carrier", CAST (FLOOR("CRSDepTime"%2400/100) AS INT) AS "Hour",
           CAST(AVG("ArrDelay") AS DECIMAL(8,2)) AS "PredictedArrDelay"
    FROM ontime
    WHERE Carrier = 'EA'
    GROUP BY "Carrier", "Hour"
),
t2 AS (
    SELECT t."Carrier", tmp.*
    FROM tmp, (SELECT DISTINCT "Carrier" FROM t1) AS t
)
SELECT "Carrier", "Hour", SUM("PredictedArrDelay")
FROM (SELECT * FROM t1 UNION SELECT * FROM t2) AS t
GROUP BY "Carrier", "Hour"
ORDER BY "Carrier", "Hour";
\end{lstlisting}
  \caption{Query 15}
  \label{sel:sql15}
\end{figure}


\begin{figure}[htb!]
  \begin{subfigure}[t]{0.5\textwidth}
    \includegraphics[scale=0.4]{figs/airtraffic/airtraffic_sel15_1_error.pdf}
    \caption{Query 15 select error}
    \label{fig:sel19}
  \end{subfigure}
  \begin{subfigure}[t]{0.5\textwidth}
    \includegraphics[scale=0.4]{figs/airtraffic/airtraffic_q15_1_memerror.pdf}
    \caption{Query 15 memory footprint}
    \label{fig:sel19}
   \end{subfigure}

\end{figure}

\subsubsection{Q19}

\begin{figure}[htb!]
\begin{lstlisting}[frame=single]
SELECT CAST (FLOOR("CRSDepTime"%2400/100) AS INT) AS "Hour",
       "Origin", "Dest", "Carrier",
       CAST(SUM("DepDel15") AS DOUBLE)/COUNT(*) >= 0.5 AS "PossibleLongDelay"
FROM ontime
WHERE Carrier = 'EV'
GROUP BY "Origin", "Dest", "Carrier", "Hour"
ORDER BY "PossibleLongDelay", "Hour", "Origin", "Dest", "Carrier";
\end{lstlisting}
  \caption{Query 19}
  \label{sel:sql19}
\end{figure}


\begin{figure}[t!]
  \begin{subfigure}[t]{0.5\textwidth}
    \includegraphics[scale=0.4]{figs/airtraffic/airtraffic_sel19_1_error.pdf}
    \caption{Query 19 select error}
    \label{fig:sel19}
  \end{subfigure}
  \begin{subfigure}[t]{0.5\textwidth}
    \includegraphics[scale=0.4]{figs/airtraffic/airtraffic_q19_1_memerror.pdf}
    \caption{Query 19 memory footprint}
    \label{fig:sel19}
   \end{subfigure}

\end{figure}



% \begin{figure}[ht]
%   \centering
%   \includegraphics[scale=0.7]{figs/airtraffic/airtraffic_q19_1_memerror.pdf}
%   \caption{Query 4 memory footprint}
%   \label{fig:sel6}
% \end{figure}

%todo list query


\end{document}
